\documentclass[10pt,conference]{IEEEtran}

\usepackage{amsmath,amssymb}
\usepackage{graphicx}
\usepackage{booktabs}
\usepackage[hyphens]{url}
\usepackage{hyperref}
\hypersetup{breaklinks=true}

\title{Regime-Aware Risk and Portfolio Allocation\\
in the S\&P 500 (H1 2025)}

\author{
\IEEEauthorblockN{Ryan Luo and Peter Quawas}
\IEEEauthorblockA{Department of Electrical and Computer Engineering\\
University of California San Diego}
}

\date{} % IEEEtran ignores \date, but include an empty one just in case

\begin{document}
\maketitle

\begin{abstract}
This project analyzes daily trade data for S\&P 500 companies during the
first half of 2025 to study how risk and return vary across time and sectors.
We construct an equal weight pseudo index from individual stock returns and
apply time series diagnostics, GARCH volatility modeling, and a two state
Markov switching model to identify high and low volatility regimes. Sector
tail risk is measured using historical Value at Risk (VaR) and Expected
Shortfall (ES). We compare equal weight, inverse volatility, and minimum
variance portfolios using a shrinkage covariance estimator. The results show
clear volatility clustering and distinct regimes, with high volatility periods
aligned with drawdowns. Portfolios that respond to risk achieve shallower
drawdowns and better risk adjusted performance, which illustrates how a
regime based allocation can support more robust investment decisions.
\end{abstract}

\section{Introduction}

Equity market volatility is not constant. Calm periods are often  interrupted by
episodes of stress that can cause concentrated portfolio losses. Standard allocations
that ignore time varying risk may be suboptimal when volatility regimes differ
sharply. This project addresses three questions for January through June 2025:
\begin{enumerate}
  \item How strong are volatility regimes in the S\&P 500?
  \item How does tail risk differ across sectors?
  \item Can simple allocation rules based on risk, such as inverse volatility
        and minimum variance portfolios, improve risk adjusted performance
        relative to equal weight portfolios?
\end{enumerate}

Understanding volatility regimes and sector tail risk is important for
position sizing, risk budgeting, and capital allocation. GARCH type models
and Markov switching regimes are standard tools for capturing volatility
clustering and regime changes~\cite{hamilton1989,bollerslev1986}. Combining
these with portfolio optimization and covariance shrinkage~\cite{ledoitwolf2004}
gives a practical example of how statistically informed risk models can
support portfolio construction in noisy, finite sample settings.

\section{Data Description and Preprocessing}

\subsection{Dataset}

We use the Kaggle dataset ``S\&P 500 Stocks Trade Data for First 6 Months of
2025,'' which contains daily OHLCV fields (open, high, low, close, adjusted
close, volume) for S\&P 500 constituents from early January through June 30,
2025. Each row is a ticker day observation. Some days such as holidays are
missing uniformly, but the panel is approximately balanced over this horizon.

\subsection{Preprocessing Steps}

We pivot the data into a wide date by ticker matrix of adjusted close prices
and compute log returns
\begin{equation}
r_{t,i} = \log P_{t,i} - \log P_{t-1,i},
\end{equation}
for stock $i$ on day $t$. The first observation per ticker is dropped, which
yields a balanced panel of daily returns. An equal weight pseudo index is the
cross sectional average of ticker returns each day.

Additional preprocessing includes alignment of trading days and removal of
rows with incomplete cross sectional coverage, winsorization or removal of
obvious data errors such as extreme returns beyond realistic market moves,
and mapping each ticker to a sector using a public S\&P 500 constituents
file, with ``Unknown'' assigned when the mapping is unavailable. The result is
a clean panel of daily log returns with sector labels.

\section{Methods}

\subsection{Time Series Diagnostics}

We analyze the equal weight index returns using a histogram with an overlaid
Normal density and a QQ plot to assess deviations from normality.
Autocorrelation functions of returns and squared returns reveal serial
dependence and volatility clustering. As is typical for equity indexes, raw
return autocorrelation is weak, while squared returns show strong positive
autocorrelation, which motivates conditional heteroskedasticity models.

\subsection{GARCH(1,1) with Student-$t$ Innovations}

We fit a GARCH(1,1) model with Student-$t$ innovations using the \texttt{arch}
Python library~\cite{archdocs}:
\begin{align}
r_t &= \mu + \varepsilon_t,\quad \varepsilon_t = \sigma_t z_t, \label{eq:garch_mean}\\
\sigma_t^2 &= \omega + \alpha_1 \varepsilon_{t-1}^2 + \beta_1 \sigma_{t-1}^2, \label{eq:garch_var}
\end{align}
where $z_t$ follows a standardized Student-$t$ distribution. The conditional
variance depends on one lag of the squared shock and one lag of previous
variance. The Student-$t$ distribution accommodates heavy tailed shocks better
than Gaussian innovations. We examine the persistence $\alpha_1 + \beta_1$ and
the conditional volatility series $\hat{\sigma}_t$.

\subsection{Markov Switching Volatility Regimes}

To identify discrete volatility regimes, we estimate a two state Markov
switching model~\cite{hamilton1989} using \texttt{statsmodels}~\cite{statsmodels}.
Let $S_t \in \{1,2\}$ denote the latent regime:
\begin{equation}
r_t = \mu + \varepsilon_t,\quad
\varepsilon_t \mid S_t = s \sim \mathcal{N}(0,\sigma_s^2).
\end{equation}
The regime evolves as a first order Markov chain with transition probabilities
$P_{ij} = \Pr(S_t = j \mid S_{t-1} = i)$. Maximum likelihood yields
regime specific variances and transition probabilities. Smoothed probabilities
$\Pr(S_t = \text{high vol} \mid \text{data})$ define a time series of regime
labels, which we overlay on the pseudo index level to compare portfolio
performance across calm and turbulent periods.

\subsection{Sector Tail Risk: VaR and ES}

For sector level risk, daily sector returns are cross sectional averages of
constituent returns. For sector $k$, we compute historical 95\% Value at Risk
and Expected Shortfall:
\begin{align}
\text{VaR}_{0.95}^k &= q_{0.05}(r^k), \label{eq:var}\\
\text{ES}_{0.95}^k &= \mathbb{E}[r^k \mid r^k \le q_{0.05}(r^k)], \label{eq:es}
\end{align}
where $q_{0.05}(r^k)$ is the 5th percentile of sector $k$'s return distribution.
The empirical breach rate, defined as the fraction of days with
$r_t^k \le \widehat{\text{VaR}}_{0.95}^k$, serves as a simple VaR backtest.

\subsection{Portfolio Construction}

Using the ticker by day returns matrix, we form three daily rebalanced
long only portfolios:
\begin{itemize}
  \item \textbf{Equal weight (EW):} $w_i = 1/N$ for $N$ stocks.
  \item \textbf{Inverse volatility (IV):} $w_i \propto 1/\hat{\sigma}_i$, based
        on each stock's return standard deviation.
  \item \textbf{Minimum variance (MV):} weights that minimize $w^\top \Sigma w$
        subject to $w^\top \mathbf{1}=1$, $w \ge 0$, where $\Sigma$ is a
        Ledoit Wolf shrinkage covariance estimate~\cite{ledoitwolf2004}.
\end{itemize}
For each portfolio, we compute daily returns, annualized return and volatility,
a Sharpe ratio that assumes a zero risk free rate, and maximum drawdown.
Metrics are also computed conditional on high and low volatility regimes.

\section{Results and Visualization}

\subsection{Diagnostics and Volatility Clustering}

Fig.~\ref{fig:diagnostics} displays the equal weight index histogram, QQ
plot, and autocorrelation functions of returns and squared returns. The
empirical distribution has heavier tails than a Normal fit, and the
autocorrelation of squared returns is significantly positive over multiple
lags. These patterns confirm volatility clustering and support the use of
conditional volatility models.

\begin{figure}[t]
  \centering
  \includegraphics[width=\linewidth]{figures/returns_diagnostics.pdf}
  \caption{Equal weight index diagnostics: histogram versus Normal fit (top left),
           Q Q plot (top right), autocorrelation of returns (bottom left), and
           autocorrelation of squared returns (bottom right). Heavy tails and
           positive autocorrelation in squared returns are evident.}
  \label{fig:diagnostics}
\end{figure}

\subsection{GARCH Results}

The GARCH(1,1) fit yields persistent conditional variance with
$\alpha_1 + \beta_1$ close to but below one, which indicates high volatility
persistence. The Student-$t$ degrees of freedom parameter is finite, which
captures heavy tails. Fig.~\ref{fig:garch} plots returns alongside GARCH
conditional volatility, and volatility spikes line up with visually turbulent
periods.

\begin{figure}[t]
  \centering
  \includegraphics[width=\linewidth]{figures/garch_volatility.pdf}
  \caption{Student-$t$ GARCH(1,1) conditional volatility (shaded) for the
           equal weight index. Spikes correspond to periods of elevated market
           stress.}
  \label{fig:garch}
\end{figure}

\subsection{Markov Switching Regimes}

Fig.~\ref{fig:regimes} overlays the pseudo index level with shaded
high volatility regimes and shows smoothed regime probabilities. The two
regimes differ substantially in variance, and high volatility windows coincide
with drawdowns. This shows that a discrete regime framework can capture
meaningful shifts in market risk.

\begin{figure}[t]
  \centering
  \includegraphics[width=\linewidth]{figures/regime_probabilities.pdf}
  \caption{Pseudo index level with high volatility regimes shaded (top) and
           smoothed probability of the high volatility regime (bottom). Regime
           switches align with drawdown periods.}
  \label{fig:regimes}
\end{figure}

\subsection{Sector VaR and ES}

Fig.~\ref{fig:sectorvar} reports historical 95\% VaR and ES by sector. Over
H1 2025, the aggregate sector has a VaR between 1\% and 2\% daily loss and an
ES around 3\% in the worst 5\% of days, with breach frequencies close to the
nominal 5\%. More cyclical sectors show more negative VaR and ES, which
indicates larger expected losses on extreme down days.

\begin{figure}[t]
  \centering
  \includegraphics[width=\linewidth]{figures/sector_var_es.pdf}
  \caption{Historical 95\% VaR and ES by sector. Cyclical sectors show
           greater tail risk than defensive sectors.}
  \label{fig:sectorvar}
\end{figure}

\subsection{Portfolio Performance}

Table~\ref{tab:portfolios} summarizes performance metrics for the three
portfolios. Inverse volatility delivers higher annualized return and lower
volatility than equal weight, which improves the return to risk tradeoff. The
minimum variance portfolio is the most defensive, with the lowest volatility
and intermediate returns. Both IV and MV have smaller maximum drawdowns than
EW.

\begin{table}[t]
  \centering
  \caption{Portfolio Metrics (Jan--Jun 2025)}
  \label{tab:portfolios}
  \begin{tabular}{lcccc}
    \toprule
    Portfolio & Ann.\ Ret. & Ann.\ Vol. & Sharpe & Max DD \\
    \midrule
    EW & 1.2\% & 22.5\% & 0.05 & $-$17.7\% \\
    IV & 3.5\% & 19.9\% & 0.17 & $-$14.5\% \\
    MV & 0.9\% & 18.7\% & 0.05 & $-$15.6\% \\
    \bottomrule
  \end{tabular}
\end{table}

Fig.~\ref{fig:portfolios} plots cumulative returns and drawdowns for the three
strategies with regime shading. The risk sensitive portfolios keep better
downside protection when volatility is elevated, which supports the value of
allocations that respond to risk during stress periods.

\begin{figure}[t]
  \centering
  \includegraphics[width=\linewidth]{figures/portfolio_performance.pdf}
  \caption{Cumulative returns (top) and drawdowns (bottom) for EW, IV, and MV
           portfolios. Shaded regions indicate high volatility regimes. IV and
           MV show smaller drawdowns during turbulent periods.}
  \label{fig:portfolios}
\end{figure}

\section{Implications and Conclusion}

\subsection{Key Findings}

The analysis shows that S\&P 500 risk over H1 2025 is highly time varying,
with distinct high and low volatility regimes. Heavy tails and volatility
clustering break simple i.i.d.\ assumptions and support the use of conditional
volatility and regime models. From a portfolio perspective, inverse volatility
and minimum variance strategies with covariance shrinkage reach more
favorable risk adjusted performance and smaller drawdowns than a naive
equal weight allocation, especially when volatility is high. Sector level VaR
and ES show that tail risk is concentrated in certain sectors, which can guide
risk budgeting and sector tilts.

\subsection{Practical Implications}

For practitioners, these results suggest that including volatility regime
information and covariance shrinkage in portfolio construction can improve
robustness without excessively complex optimization. Simple rules such as inverse
volatility weighting already give meaningful drawdown reduction. Monitoring
regime probabilities can also inform dynamic position sizing and hedging
decisions.

\subsection{Limitations and Future Work}

Limitations include the short six month horizon of our dataset, construction of an equal
weight index rather than a cap weighted index, omission of transaction costs
and turnover constraints, and reliance on a two state regime specification.
Future work could extend the horizon, add macro predictors to regime
probabilities, explore alternative volatility models such as EGARCH or
GJR GARCH, and implement explicit regime based allocation rules that scale
exposure based on inferred risk states.

\section*{Appendix: Code}

A complete Python implementation is provided in the accompanying Jupyter
notebook, which loads the Kaggle dataset, performs preprocessing, estimates
GARCH and Markov switching models using the \texttt{arch}~\cite{archdocs} and
\texttt{statsmodels}~\cite{statsmodels} libraries, computes sector VaR and ES,
and constructs the three portfolios.

\begin{thebibliography}{9}

\bibitem{hamilton1989}
J.~D. Hamilton,
``A new approach to the economic analysis of nonstationary time series and the
business cycle,''
\emph{Econometrica}, vol.~57, no.~2, pp.~357--384, 1989.

\bibitem{bollerslev1986}
T.~Bollerslev,
``Generalized autoregressive conditional heteroskedasticity,''
\emph{J. Econometrics}, vol.~31, no.~3, pp.~307--327, 1986.

\bibitem{ledoitwolf2004}
O.~Ledoit and M.~Wolf,
``Honey, I shrunk the sample covariance matrix,''
\emph{J. Portfolio Manage.}, vol.~30, no.~4, pp.~110--119, 2004.

\bibitem{archdocs}
K.~Sheppard,
``arch: Autoregressive conditional heteroskedasticity models in Python,''
\url{https://arch.readthedocs.io/}, accessed Dec.~2025.

\bibitem{statsmodels}
S.~Seabold and J.~Perktold,
``Statsmodels: Econometric and statistical modeling with Python,''
in \emph{Proc. 9th Python Sci. Conf.}, 2010.

\end{thebibliography}

\end{document}