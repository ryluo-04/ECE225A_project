\documentclass[10pt,twocolumn]{article}
\usepackage[margin=1in]{geometry}
\usepackage{amsmath,amssymb}
\usepackage{graphicx}
\usepackage{booktabs}
\usepackage[hyphens]{url}
\usepackage{hyperref}
\hypersetup{breaklinks=true}

\title{Regime-Aware Risk and Portfolio Allocation\\
in the S\&P 500 (H1 2025)}

\author{Ryan Luo and Peter Quawas\\
ECE 225A}
\date{December 2025}

\begin{document}
\maketitle

\begin{abstract}
This project analyzes daily trade data for S\&P 500 constituents during the
first half of 2025 to study how risk and return vary across time and sectors.
We construct an equal-weight ``pseudo index'' from individual stock returns and
apply time-series diagnostics, GARCH volatility modeling, and a two-state
Markov switching model to identify high- and low-volatility regimes. Sector
tail risk is quantified via historical Value-at-Risk (VaR) and Expected
Shortfall (ES), and we compare equal-weight, inverse-volatility, and
minimum-variance portfolios using a shrinkage covariance estimator. Results
show clear volatility clustering and distinct regimes, with high-volatility
periods aligned with drawdowns. Inverse-volatility and minimum-variance
portfolios achieve shallower drawdowns and improved risk-adjusted performance
relative to equal-weight, illustrating how regime-aware, risk-sensitive
allocation can support more robust investment decisions.
\end{abstract}

\section{Problem Addressed and Significance}

\textbf{Problem addressed.}
Equity market volatility is not constant over time: tranquil periods are
interrupted by episodes of stress that concentrate portfolio losses.
Standard portfolio allocations that ignore time variation in risk may be
suboptimal when volatility regimes differ sharply. This project asks, for
January--June 2025: (i) how strong are volatility regimes in the S\&P 500,
(ii) how does tail risk differ across sectors, and (iii) can simple
risk-aware allocation rules---inverse-volatility and minimum-variance---improve
risk-adjusted performance relative to equal-weight portfolios.

\textbf{Significance.}
Understanding volatility regimes and sectoral tail risk is important for
position sizing, risk budgeting, and capital allocation. GARCH-type models
and Markov switching regimes are standard tools for capturing volatility
clustering and regime changes~\cite{hamilton1989,bollerslev1986}. Combining
these with portfolio optimization and covariance shrinkage
\cite{ledoitwolf2004} provides a practical example of how statistically
informed risk models can support portfolio construction in noisy,
finite-sample settings.

\section{Data and Preprocessing}

\subsection{Dataset}

We use the Kaggle dataset ``S\&P 500 Stocks Trade Data for First 6 Months of
2025,'' which contains daily OHLCV fields (open, high, low, close, adjusted
close, volume) for S\&P 500 constituents from early January through
June 30, 2025. Each row corresponds to a ticker-day observation. Some days
(e.g., holidays) are missing uniformly across tickers, but the panel is close
to balanced over this horizon.

We pivot the data to a wide date-by-ticker matrix of adjusted close prices,
then compute log returns
\[
r_{t,i} = \log P_{t,i} - \log P_{t-1,i}
\]
for stock $i$ on day $t$. The first observation per ticker is dropped, leaving
a balanced panel of daily returns. An equal-weight pseudo index is defined as
the cross-sectional average of ticker returns on each day.

\subsection{Preprocessing}

Preprocessing steps include:
\begin{itemize}
  \item alignment of trading days across tickers and removal of rows with
        incomplete cross-sectional coverage;
  \item removal or winsorization of obvious data errors (extremely large
        returns far beyond realistic market moves);
  \item mapping each ticker to a sector using a public S\&P 500 constituents
        file when accessible, with a fallback ``Unknown'' sector when not.
\end{itemize}
The result is a clean panel of daily log returns with sector labels for
sector-level analysis.

\section{Methods and Ideas}

\subsection{Time-Series Diagnostics}

We first analyze the equal-weight index returns. A histogram with overlaid
Normal density and a Q--Q plot assess deviations from normality. Sample
autocorrelation functions (ACF) of returns and squared returns reveal whether
there is serial correlation and volatility clustering. As expected for equity
indexes, raw return autocorrelation is weak, while squared returns exhibit
strong positive autocorrelation, motivating conditional heteroskedasticity
models.

\subsection{GARCH(1,1) Volatility Model}

We fit a GARCH(1,1) model with Student-\emph{t} innovations to the index
returns using the \texttt{arch} Python library~\cite{archdocs}. The model is
\begin{align}
r_t &= \mu + \varepsilon_t,\quad \varepsilon_t = \sigma_t z_t, \\
\sigma_t^2 &= \omega + \alpha_1 \varepsilon_{t-1}^2
                    + \beta_1 \sigma_{t-1}^2,
\end{align}
where $z_t$ follows a standardized Student-\emph{t} distribution. The
(1,1) order means the conditional variance today depends on one lag of the
squared shock and one lag of the previous variance, while the Student-\emph{t}
distribution allows for heavy-tailed shocks, which better match equity
returns than Gaussian innovations. We examine the persistence
$\alpha_1 + \beta_1$ and the conditional volatility series
$\hat{\sigma}_t$.

\subsection{Markov Switching Volatility Regimes}

To identify discrete volatility regimes, we estimate a two-state Markov
switching model for the index returns~\cite{hamilton1989}. Let $S_t \in
\{1,2\}$ be the latent regime and assume
\[
r_t = \mu + \varepsilon_t,\quad
\varepsilon_t \mid S_t = s \sim \mathcal{N}(0,\sigma_s^2).
\]
The regime evolves as a first-order Markov chain with transition
probabilities $P_{ij} = \Pr(S_t = j \mid S_{t-1} = i)$. Maximum likelihood
estimation yields regime-specific variances and transition probabilities.
Smoothed probabilities $\Pr(S_t = \text{high-vol} \mid \text{data})$ define a
time series of regime labels, which we overlay on the pseudo index level and
use to compare portfolio performance across calm versus turbulent periods.

\subsection{Sector Tail Risk: VaR and ES}

For sector-level risk, we compute daily sector returns as cross-sectional
averages of constituent returns by sector and date. For each sector $k$, we
estimate historical 95\% Value-at-Risk (VaR) and Expected Shortfall (ES):
\begin{align}
\text{VaR}_{0.95}^k &= q_{0.05}(r^k),\\
\text{ES}_{0.95}^k &= \mathbb{E}[r^k \mid r^k \le q_{0.05}(r^k)],
\end{align}
where $q_{0.05}(r^k)$ is the 5th percentile of sector $k$'s return
distribution. We also compute the empirical breach rate, the fraction of days
with $r_t^k \le \widehat{\text{VaR}}_{0.95}^k$, as a simple VaR backtest.

\subsection{Portfolio Strategies and Covariance Shrinkage}

Using the full ticker-by-day returns matrix, we form three daily-rebalanced
long-only portfolios:
\begin{itemize}
  \item \textbf{Equal-weight (EW)}: $w_i = 1/N$ for $N$ stocks.
  \item \textbf{Inverse-volatility (IV)}: $w_i \propto 1/\hat{\sigma}_i$
        based on each stock's return standard deviation.
  \item \textbf{Minimum-variance (MV)}: weights minimizing
        $w^\top \Sigma w$ subject to $w^\top \mathbf{1}=1$, $w \ge 0$, where
        $\Sigma$ is a Ledoit--Wolf shrinkage covariance estimate
        \cite{ledoitwolf2004}.
\end{itemize}
For each portfolio, we compute daily returns, annualized return and
volatility, a Sharpe-like ratio (assuming zero risk-free rate), and maximum
drawdown. We also recompute these metrics conditional on high- and low-vol
regimes identified by the Markov switching model.

\section{Results and Visualization}

\subsection{Diagnostics and Volatility Clustering}

Figure~\ref{fig:diagnostics} shows the equal-weight index histogram, Q--Q
plot, and ACFs of returns and squared returns. The empirical distribution has
heavier tails than a Normal fit, and the squared-return ACF is significantly
positive over multiple lags, confirming volatility clustering and validating
the use of conditional volatility models.

\begin{figure}[t]
  \centering
  \includegraphics[width=\linewidth]{figures/returns_diagnostics.pdf}
  \caption{Equal-weight index diagnostics: histogram vs Normal fit,
           Q--Q plot, and ACFs of returns and squared returns.}
  \label{fig:diagnostics}
\end{figure}

\subsection{GARCH Volatility and Regimes}

The GARCH(1,1) fit yields persistent conditional variance with
$\alpha_1 + \beta_1$ close to but below one, and a finite Student-\emph{t}
degrees-of-freedom parameter capturing heavy tails. Figure~\ref{fig:garch}
plots the pseudo index returns and GARCH conditional volatility; volatility
spikes align with visually turbulent periods.

\begin{figure}[t]
  \centering
  \includegraphics[width=\linewidth]{figures/garch_volatility.pdf}
  \caption{Student-\emph{t} GARCH(1,1) conditional volatility for the
           equal-weight index.}
  \label{fig:garch}
\end{figure}

Figure~\ref{fig:regimes} overlays the pseudo index level with shaded
high-volatility regimes from the Markov switching model and shows smoothed
probabilities of the high-volatility regime. The two regimes differ
substantially in variance, and high-volatility windows coincide with
drawdowns.

\begin{figure}[t]
  \centering
  \includegraphics[width=\linewidth]{figures/regime_probabilities.pdf}
  \caption{Pseudo index level with high-volatility regimes shaded (top) and
           smoothed probability of the high-volatility regime (bottom).}
  \label{fig:regimes}
\end{figure}

\subsection{Sector VaR and ES}

Figure~\ref{fig:sectorvar} reports historical 95\% VaR and ES by sector (or
aggregate if sector labels are unavailable). Over H1 2025, the aggregate
sector exhibits a VaR on the order of a 1--2\% daily loss and an ES around
3\% in the worst 5\% of days, with breach frequencies near the nominal 5\%.
More cyclical sectors naturally show more negative VaR and ES, indicating
larger expected losses on extreme down days.

\begin{figure}[t]
  \centering
  \includegraphics[width=\linewidth]{figures/sector_var_es.pdf}
  \caption{Historical 95\% VaR and ES by sector (or aggregate).}
  \label{fig:sectorvar}
\end{figure}

\subsection{Portfolio Performance}

Table~\ref{tab:portfolios} summarizes performance metrics; values should be
filled in from the notebook output. In our runs, inverse-volatility delivers
higher return and lower volatility than equal-weight, improving the
return-to-risk tradeoff. The minimum-variance portfolio is the most defensive,
with the lowest volatility and intermediate returns. Both IV and MV show
shallower maximum drawdowns than EW.

\begin{table}[t]
  \centering
  \caption{Portfolio metrics (Jan--Jun 2025).}
  \label{tab:portfolios}
  \begin{tabular}{lcccc}
    \toprule
    Portfolio & Ann.\ Ret. & Ann.\ Vol. & Sharpe & Max DD \\
    \midrule
    EW & 1.2\% & 22.5\% & 0.05 & -17.7\% \\
    IV & 3.5\% & 19.9\% & 0.17 & -14.5\% \\
    MV & 0.9\% & 18.7\% & 0.05 & -15.6\% \\
    \bottomrule
  \end{tabular}
\end{table}

Figure~\ref{fig:portfolios} plots cumulative returns and drawdowns for the
three strategies with regime shading. Risk-sensitive portfolios retain better
downside protection when volatility is elevated.

\begin{figure}[t]
  \centering
  \includegraphics[width=\linewidth]{figures/portfolio_performance.pdf}
  \caption{Cumulative returns and drawdowns for EW, IV, and MV portfolios
           with volatility regime shading.}
  \label{fig:portfolios}
\end{figure}

\section{Implications and Conclusion}

\textbf{Implications.}
The analysis confirms that S\&P 500 risk over H1 2025 is highly time-varying,
with distinct high- and low-volatility regimes. Heavy tails and volatility
clustering violate simple i.i.d.\ assumptions and justify conditional
volatility and regime models. From a portfolio perspective,
inverse-volatility and shrinkage-based minimum-variance strategies achieve
more favorable risk-adjusted performance and reduced drawdowns relative to
naive equal-weight allocation, particularly when volatility is elevated.
Sector-level VaR and ES further illustrate that tail risk is concentrated in
certain sectors, which can guide risk budgeting and sector tilts.

\textbf{Limitations and future work.}
Limitations include the short six-month horizon, equal-weight index
construction rather than cap-weighting, absence of transaction costs or
turnover constraints, and reliance on a two-state regime specification.
Future work could extend the horizon, incorporate macro predictors in the
regime probabilities, explore alternative volatility models (e.g., EGARCH or
GJR-GARCH), and implement explicit regime-aware allocation rules that scale
exposure based on inferred risk states.

\section*{Appendix: Code}

A complete Python implementation is provided in the accompanying Jupyter
notebook, which loads the Kaggle dataset, performs preprocessing, estimates
the GARCH and Markov switching models, computes sector VaR/ES, and constructs
the three portfolios. Selected code snippets can be included here as
verbatim listings if required.

\begin{thebibliography}{9}

\bibitem{hamilton1989}
J.~D. Hamilton,
``A New Approach to the Economic Analysis of Nonstationary Time Series and the
Business Cycle,''
\emph{Econometrica}, vol.~57, no.~2, pp. 357--384, 1989.

\bibitem{bollerslev1986}
T.~Bollerslev,
``Generalized Autoregressive Conditional Heteroskedasticity,''
\emph{Journal of Econometrics}, vol.~31, no.~3, pp. 307--327, 1986.

\bibitem{ledoitwolf2004}
O.~Ledoit and M.~Wolf,
``Honey, I Shrunk the Sample Covariance Matrix,''
\emph{Journal of Portfolio Management}, vol.~30, no.~4, pp. 110--119, 2004.

\bibitem{archdocs}
K.~Sheppard,
\emph{arch: Autoregressive Conditional Heteroskedasticity models in Python},
\url{https://arch.readthedocs.io/}, accessed Dec.~2025.

\bibitem{statsmodels}
S.~Seabold and J.~Perktold,
``Statsmodels: Econometric and Statistical Modeling with Python,''
in \emph{Proceedings of the 9th Python in Science Conference}, 2010.

\end{thebibliography}

\end{document}